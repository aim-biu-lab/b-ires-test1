\documentclass[12pt, a4paper]{article}

% Packages
\usepackage[utf8]{inputenc}
\usepackage[T1]{fontenc}
\usepackage{lmodern}
\usepackage[margin=1in]{geometry}
\usepackage{graphicx}
\usepackage{hyperref}
\usepackage{enumitem}
\usepackage{booktabs}
\usepackage{xcolor}
\usepackage{titlesec}
\usepackage{fancyhdr}
\usepackage{microtype}
\usepackage{listings}
\usepackage{longtable}

% Color definitions
\definecolor{primaryblue}{RGB}{31, 78, 121}
\definecolor{accentgray}{RGB}{89, 89, 89}
\definecolor{codebg}{RGB}{245, 245, 245}
\definecolor{codeframe}{RGB}{200, 200, 200}

% Hyperlink styling
\hypersetup{
    colorlinks=true,
    linkcolor=primaryblue,
    urlcolor=primaryblue,
    citecolor=primaryblue
}

% Code listing style
\lstset{
    backgroundcolor=\color{codebg},
    frame=single,
    rulecolor=\color{codeframe},
    basicstyle=\ttfamily\small,
    breaklines=true,
    columns=fullflexible,
    keepspaces=true,
    showstringspaces=false
}

% Header/Footer
\pagestyle{fancy}
\fancyhf{}
\fancyhead[L]{\textcolor{accentgray}{\small B-IRES Platform}}
\fancyhead[R]{\textcolor{accentgray}{\small Platform Overview}}
\fancyfoot[C]{\thepage}

% Title formatting
\titleformat{\section}
{\normalfont\Large\bfseries\color{primaryblue}}{\thesection}{1em}{}
\titleformat{\subsection}
{\normalfont\large\bfseries\color{primaryblue}}{\thesubsection}{1em}{}
\titleformat{\subsubsection}
{\normalfont\normalsize\bfseries\color{primaryblue}}{\thesubsubsection}{1em}{}

% Document info
\title{
    \vspace{-1cm}
    \textbf{\Huge B-IRES}\\[0.3em]
    \Large Bar-Ilan Research Evaluation System\\[0.5em]
    \large A Comprehensive Platform Overview
}
\author{B-IRES Development Team}
\date{\today}

\begin{document}

\maketitle

\begin{abstract}
The \textbf{Bar-Ilan Research Evaluation System (B-IRES)} is a configuration-driven platform designed to simplify the creation, deployment, and management of multi-modal research trials and human-centric evaluations. Built with researchers in mind, B-IRES enables non-technical users to design sophisticated multi-stage studies through intuitive YAML configuration files, while providing robust data collection, real-time logging, and seamless participant management. 

The platform features a \textbf{backend-authoritative architecture} ensuring data integrity, \textbf{offline-capable participant sessions} with automatic synchronization, \textbf{role-based access control} for collaborative research teams, and a \textbf{pluggable storage system} supporting multiple database backends. Current capabilities include a professional YAML editor with live validation and preview, comprehensive block components for diverse content types, and real-time monitoring dashboards.

Future development roadmap includes a \textbf{Visual Experiment Builder} with drag-and-drop stage ordering, visual flow editing for conditional branching, and one-click YAML generation---enabling researchers with no coding experience to construct complex studies visually. Additionally, enhanced \textbf{Live Monitoring} capabilities will provide active session tracking, real-time quota status, error logging, and advanced analytics including completion rates, response time distributions, and drop-off analysis.

This document presents a high-level overview of the platform's capabilities, features, and potential applications across various research domains.
\end{abstract}

\vspace{1em}
\hrule
\vspace{2em}

%========================================
\section{Introduction}
%========================================

Research involving human participants---whether in psychology, cognitive science, human-computer interaction, or social sciences---often requires complex experimental setups involving multiple stages, conditional logic, multimedia content, and detailed data logging. Traditionally, creating such multi-modal research trials demanded significant technical expertise in web development, database management, and server administration---skills that many researchers lack.

\textbf{B-IRES} (Bar-Ilan Research Evaluation System) addresses this gap by providing a unified platform that empowers researchers to:

\begin{itemize}[leftmargin=*]
    \item Design human-centric evaluations through simple, human-readable configuration files
    \item Deploy complete experimental environments with minimal technical knowledge
    \item Collect granular participant data in real-time with millisecond precision
    \item Manage concurrent participants reliably with atomic quota management
    \item Collaborate with team members through role-based access control (Admin, Researcher, Viewer)
    \item Ensure data integrity through backend-authoritative state management
    \item Support offline participant sessions with automatic synchronization upon reconnection
\end{itemize}

The platform philosophy centers on the principle that researchers should focus on \textit{what} they want to study, not \textit{how} to implement the technical infrastructure.

\subsection{Platform Roadmap}

B-IRES is under active development with an ambitious feature roadmap:

\begin{itemize}[leftmargin=*]
    \item \textbf{Visual Experiment Builder} (Planned): A drag-and-drop interface for constructing studies without writing YAML. Features will include visual flow editors showing conditional branching, component palettes with pre-built blocks, property panels for configuration, and live preview. The builder will generate YAML automatically, with an ``eject to code'' option for advanced customization.
    
    \item \textbf{Advanced Live Monitoring} (Planned): Real-time dashboards showing active sessions, quota status across conditions, error logs, and participant progress. Analytics will include completion rate tracking, response time distributions, drop-off analysis by stage, and anomaly detection.
    
    \item \textbf{Experiment Version Control} (Implemented): Every configuration save creates a new version, enabling rollback, comparison, and audit trails. Published experiments are immutable snapshots ensuring reproducibility.
    
    \item \textbf{Semantic Theme System} (Implemented): Pre-built visual themes (clinical, academic, high-contrast, dark mode) that researchers select by name, ensuring consistent branding without CSS knowledge.
\end{itemize}

%========================================
\section{Vision and Philosophy}
%========================================

B-IRES is built on four foundational principles:

\subsection{Accessibility First}
The platform eliminates technical barriers. A researcher with no programming experience should be able to configure and deploy a sophisticated human-centric evaluation by editing straightforward text files and running a single command. The system handles all underlying complexity---from database setup to user session management.

\subsection{Configuration Over Code}
Every aspect of a study---stages, questions, branching logic, timing constraints, and visual elements---is defined in declarative YAML files. This \textit{Infrastructure-as-Configuration} approach means research trials are:
\begin{itemize}[leftmargin=*]
    \item Reproducible and version-controllable
    \item Easy to modify without touching source code
    \item Shareable between researchers and institutions
    \item Immutable once published (templates are flattened into snapshots)
\end{itemize}

\subsection{Research-Grade Reliability}
Human-centric evaluations require precise timing, consistent presentation, and comprehensive logging. B-IRES ensures that every participant interaction is captured, every response is timestamped with millisecond precision, and data integrity is maintained through redundant storage mechanisms and backend-authoritative state management.

\subsection{Backend Authority}
The server is the single source of truth for session state. Participants cannot manipulate their progress or skip stages via browser developer tools. Every state transition is validated server-side, and the client maintains only a read-only mirror of the authoritative state.

%========================================
\section{Core Features}
%========================================

\subsection{Flexible Study Design}

B-IRES supports a hierarchical experiment structure organized as:

\begin{center}
\textbf{Experiment} $\rightarrow$ \textbf{Stages} $\rightarrow$ \textbf{Sub-Stages (Blocks)}
\end{center}

Each stage can contain multiple sub-stages, and each sub-stage can be one of several types:

\begin{itemize}[leftmargin=*]
    \item \textbf{User Information}: Standardized demographic collection with validation rules
    \item \textbf{Questionnaires}: Single or multiple-choice questions, text inputs, and scaled responses
    \item \textbf{Rich Content}: HTML-formatted text, images, and styled presentations (inline or from files)
    \item \textbf{Multimedia}: Video and audio stimuli with full playback control and event logging
    \item \textbf{Interactive Tasks}: Embedded applications via sandboxed iframes with bidirectional communication
    \item \textbf{Consent Forms}: Legal consent with checkbox confirmation and forced reading time
    \item \textbf{Attention Checks}: Hidden validation questions to ensure data quality
\end{itemize}

\subsection{Conditional Branching and Dynamic Flow}

Real-world research often requires adaptive pathways based on participant responses. B-IRES's logic engine (powered by JsonLogic predicates) supports:

\begin{itemize}[leftmargin=*]
    \item \textbf{Response-Based Routing}: Show or hide stages based on previous answers
    \item \textbf{Parameter-Driven Assignment}: Assign participants to conditions based on URL parameters (e.g., recruitment source, experimental group)
    \item \textbf{Demographic Filters}: Present different content based on participant characteristics
    \item \textbf{Quota Management}: Automatically close conditions when target sample sizes are reached
\end{itemize}

\subsubsection{Example 1: Age-Based Routing}

If a participant indicates they are under 18 years old, the system can automatically skip consent procedures requiring parental approval and redirect to an age-appropriate path:

\begin{lstlisting}
stages:
  - id: "demographics"
    type: "user_info"
    fields:
      - field: "age"
        type: "number"
        required: true

  - id: "adult_consent"
    type: "consent_form"
    visibility_rule: "demographics.age >= 18"
    
  - id: "minor_notice"
    type: "content_display"
    visibility_rule: "demographics.age < 18"
    content: "This study requires participants aged 18+."
\end{lstlisting}

\subsubsection{Example 2: Experience-Based Additional Questions}

For studies involving gaming or technical tasks, you can present additional questions to experienced participants while skipping them for novices:

\begin{lstlisting}
stages:
  - id: "gaming_background"
    type: "questionnaire"
    questions:
      - id: "gaming_frequency"
        text: "How often do you play video games?"
        type: "select"
        options:
          - value: "never"
            label: "Never"
          - value: "rarely"
            label: "Rarely (few times a year)"
          - value: "monthly"
            label: "Monthly"
          - value: "weekly"
            label: "Weekly"
          - value: "daily"
            label: "Daily"

  # Only shown to frequent gamers (weekly or daily)
  - id: "gamer_details"
    type: "questionnaire"
    visibility_rule: >
      gaming_background.gaming_frequency == 'weekly' || 
      gaming_background.gaming_frequency == 'daily'
    questions:
      - id: "preferred_genre"
        text: "What is your preferred game genre?"
        type: "select"
        options:
          - value: "fps"
            label: "First-Person Shooter"
          - value: "rpg"
            label: "Role-Playing Games"
          - value: "strategy"
            label: "Strategy"
          - value: "puzzle"
            label: "Puzzle"
          - value: "sports"
            label: "Sports/Racing"
            
      - id: "hours_per_week"
        text: "How many hours per week do you typically play?"
        type: "number"
        min: 1
        max: 100
        
      - id: "competitive_gaming"
        text: "Do you participate in competitive/ranked gaming?"
        type: "radio"
        options:
          - value: "yes"
            label: "Yes, regularly"
          - value: "sometimes"
            label: "Occasionally"
          - value: "no"
            label: "No"

  # Advanced gaming task only for competitive players
  - id: "advanced_gaming_task"
    type: "iframe_sandbox"
    visibility_rule: "gamer_details.competitive_gaming == 'yes'"
    source: "/tasks/advanced_reaction_test.html"
    
  # Standard task for all others
  - id: "standard_task"
    type: "iframe_sandbox"
    visibility_rule: "gamer_details.competitive_gaming != 'yes'"
    source: "/tasks/basic_reaction_test.html"
\end{lstlisting}

\subsection{Timing and Pacing Controls}

Precise temporal control is essential for many experimental paradigms. B-IRES provides:

\begin{itemize}[leftmargin=*]
    \item \textbf{Minimum Display Time}: Ensure participants cannot skip content too quickly (e.g., consent forms must be displayed for at least 30 seconds before the ``Next'' button enables)
    \item \textbf{Maximum Response Time}: Set deadlines with configurable automatic actions (auto-submit, skip stage, or lock interface)
    \item \textbf{Visual Countdowns}: Optional timers displayed to participants
    \item \textbf{Media Synchronization}: Advance only when videos complete, pause, or reach specific timestamps
\end{itemize}

\subsection{Specialized Question Types}

Beyond standard form inputs, B-IRES includes research-specific question formats:

\begin{itemize}[leftmargin=*]
    \item \textbf{Likert Scales}: Visual representations (faces, emoticons) for 5-point or 7-point scales with configurable visual themes
    \item \textbf{Semantic Differentials}: Bipolar scales with labeled endpoints
    \item \textbf{Matrix Questions}: Multiple items rated on the same scale in a grid format
    \item \textbf{Ranking Tasks}: Drag-and-drop ordering of items
    \item \textbf{Hidden Fields}: Automatic capture of metadata (timestamp, browser info, referral source, URL parameters)
\end{itemize}

\subsection{Progress Visualization}

Participant engagement improves with clear progress indication. B-IRES automatically displays:

\begin{itemize}[leftmargin=*]
    \item Current stage and total stages (``Stage 2 of 5'')
    \item Question progress within stages (``Question 3 of 10'')
    \item Overall completion percentage with progress bar
    \item Time remaining (when applicable)
    \item Optional sidebar navigation showing completed, current, and upcoming stages
\end{itemize}

Progress calculations dynamically adjust based on conditional branching---if certain stages are skipped due to visibility rules, they don't appear in the count.

\subsection{Offline Support and Resilience}

B-IRES includes built-in resilience for unreliable network conditions:

\begin{itemize}[leftmargin=*]
    \item \textbf{State Hydration}: Session state is persisted to browser storage; if the browser crashes or refreshes, the participant resumes exactly where they left off
    \item \textbf{Offline Event Queue}: Interactions are queued locally (IndexedDB) when offline and synchronized automatically upon reconnection
    \item \textbf{Idempotent Synchronization}: Each event carries a unique idempotency key; duplicate submissions (from retries) are detected and deduplicated server-side
    \item \textbf{Visual Offline Indicator}: Participants see clear feedback when operating offline (``Will sync when connected'')
\end{itemize}

%========================================
\section{Data Collection and Logging}
%========================================

\subsection{Comprehensive Event Tracking}

B-IRES captures every meaningful participant interaction:

\begin{itemize}[leftmargin=*]
    \item \textbf{Navigation Events}: Stage entries, exits, and transitions
    \item \textbf{Response Events}: Answer selections, modifications, and submissions
    \item \textbf{Interaction Events}: Button clicks, video plays/pauses/seeks, scroll actions, fullscreen toggles
    \item \textbf{Timing Data}: Duration on each stage, response latencies, total session time
    \item \textbf{System Events}: Page visibility changes, window focus/blur, connection status, offline queue sync
\end{itemize}

Each event is timestamped with millisecond precision and tagged with unique identifiers for the participant, session, stage, and element. Events include an idempotency key for deduplication during offline sync.

\subsection{Single-Write Architecture with Async Backup}

B-IRES employs a robust storage strategy that avoids the dual-write anti-pattern:

\begin{itemize}[leftmargin=*]
    \item \textbf{Primary Database (MongoDB)}: Synchronous write for every event---fast, reliable, and immediately queryable
    \item \textbf{Object Storage Backup (S3/MinIO)}: Asynchronous background worker monitors database changes and exports to object storage as JSON Lines files
    \item \textbf{On-Demand CSV Export}: Researchers request CSV exports through the dashboard; data is transformed from MongoDB into analysis-ready formats
\end{itemize}

This architecture ensures:
\begin{itemize}[leftmargin=*]
    \item No blocking I/O on the API path
    \item No partial-write consistency issues
    \item Scalable deployment (no local filesystem dependencies)
    \item Data survival even in cases of primary database issues
\end{itemize}

\subsection{Participant Identity Management}

The platform distinguishes between:

\begin{itemize}[leftmargin=*]
    \item \textbf{User ID}: A persistent UUIDv4 stored in an HTTP-only cookie (1-year duration) that tracks the same individual across multiple sessions---useful for longitudinal studies and preventing duplicate participation
    \item \textbf{Session ID}: A unique UUIDv4 generated for each experimental run
    \item \textbf{External IDs}: URL parameters (e.g., \texttt{?worker\_id=123}) are automatically captured and linked to internal identifiers
\end{itemize}

%========================================
\section{Use Cases and Applications}
%========================================

B-IRES's flexibility makes it suitable for diverse multi-modal research scenarios:

\subsection{Psychological Research}

\begin{itemize}[leftmargin=*]
    \item \textbf{Cognitive Tasks}: Reaction time studies, memory tests, attention tasks via embedded applications
    \item \textbf{Survey Research}: Multi-scale personality assessments, attitude measurements with Likert scales
    \item \textbf{Emotional Response Studies}: Stimulus presentation (video/images) with immediate affect ratings
    \item \textbf{Decision-Making Studies}: Economic games, risk preference tasks, gambling paradigms
\end{itemize}

\subsection{Human-Computer Interaction}

\begin{itemize}[leftmargin=*]
    \item \textbf{Product Evaluation}: A/B testing with embedded prototypes
    \item \textbf{Usability Studies}: Task completion with think-aloud protocols
    \item \textbf{Feature Preference}: Conjoint analysis and feature ranking
    \item \textbf{Interface Comparison}: Side-by-side evaluation with randomized presentation order
\end{itemize}

\subsection{Educational Assessment}

\begin{itemize}[leftmargin=*]
    \item \textbf{Learning Studies}: Pre-test, intervention, post-test designs with conditional content
    \item \textbf{Instructional Design Research}: Comparing teaching method effectiveness
    \item \textbf{Adaptive Testing}: Dynamic difficulty adjustment based on performance
    \item \textbf{Knowledge Assessment}: Multi-stage examinations with attention checks
\end{itemize}

\subsection{Market Research}

\begin{itemize}[leftmargin=*]
    \item \textbf{Concept Testing}: Evaluating product ideas with target demographics
    \item \textbf{Brand Perception}: Semantic differential scales across brand attributes
    \item \textbf{Advertisement Effectiveness}: Exposure and recall studies with video stimuli
    \item \textbf{Consumer Behavior}: Choice tasks and willingness-to-pay studies
\end{itemize}

\subsection{Healthcare and Clinical Research}

\begin{itemize}[leftmargin=*]
    \item \textbf{Patient Surveys}: Symptom tracking, quality of life assessments with validated instruments
    \item \textbf{Treatment Studies}: Comparing intervention groups with proper randomization via URL parameters
    \item \textbf{Digital Therapeutics}: Delivering and measuring therapeutic content with compliance tracking
    \item \textbf{Screening Instruments}: Multi-stage diagnostic assessments with conditional follow-ups
\end{itemize}

%========================================
\section{Concurrency and Scale}
%========================================

Modern research often requires large sample sizes collected rapidly. B-IRES is designed to handle:

\begin{itemize}[leftmargin=*]
    \item \textbf{Concurrent Participants}: 100+ simultaneous users without performance degradation, thanks to asynchronous I/O and optimized database operations
    \item \textbf{Atomic Quota Management}: Condition closure when target N is reached using atomic database operations (Redis-backed reservations) preventing over-enrollment
    \item \textbf{Race Condition Prevention}: Reservation systems with Time-To-Live (TTL) ensure that if only one slot remains, exactly one participant receives it
    \item \textbf{Graceful Overflow}: Participants who cannot be assigned are automatically redirected to alternative paths, fallback stages, or completion screens
    \item \textbf{Horizontal Scaling}: Stateless backend design supports Kubernetes deployment with multiple replicas
\end{itemize}

%========================================
\section{Deployment Simplicity}
%========================================

A core B-IRES principle is that deployment should require no technical expertise beyond basic file management. The platform uses containerization technology (Docker) to package all components---web server, database, cache, and application---into a self-contained unit.

\subsection{Researcher Workflow}

\begin{enumerate}[leftmargin=*]
    \item \textbf{Configure}: Edit YAML files to define experiment structure, questions, and logic (or use the visual builder when available)
    \item \textbf{Upload Assets}: Use the Media Library to upload images, videos, and HTML files to object storage
    \item \textbf{Validate}: Run the built-in validator to check configuration syntax, asset references, and logical consistency
    \item \textbf{Launch}: Run a single command (\texttt{docker-compose up}) to start the entire platform
    \item \textbf{Distribute}: Share the experiment URL with participants
    \item \textbf{Monitor}: View real-time data through the admin dashboard
    \item \textbf{Export}: Download results in analysis-ready formats (CSV, JSON, SPSS-compatible)
\end{enumerate}

\subsection{Self-Contained Environment}

The platform runs entirely within isolated containers, meaning:

\begin{itemize}[leftmargin=*]
    \item No conflicts with existing software on the server
    \item Identical behavior across different operating systems (Linux, macOS, Windows)
    \item Easy backup and migration (export data, copy configuration, run elsewhere)
    \item Simple updates (pull new container versions)
    \item Multiple deployment options: local Docker Compose, single cloud server with Nginx/SSL, or Kubernetes for high-scale production
\end{itemize}

%========================================
\section{Integration Capabilities}
%========================================

\subsection{External Task Integration}

B-IRES can embed external applications (games, simulations, custom tasks) while maintaining centralized data collection. The embedded content communicates with the platform through a standardized messaging protocol (\texttt{window.postMessage}), allowing:

\begin{itemize}[leftmargin=*]
    \item External tasks to report events, scores, and completion status to the central log
    \item The platform to control task state (start, pause, reset) via commands
    \item Synchronized progression (advance only when the task signals completion)
    \item Context injection: the shell automatically adds user ID, session ID, stage ID, and timestamp to all logged events
\end{itemize}

This enables researchers to use existing validated tasks or develop custom stimuli while benefiting from B-IRES's infrastructure, logging, and participant management.

\subsection{Asset Management}

All user-uploaded assets (images, videos, HTML files) are stored in object storage (S3/MinIO), not on the local filesystem. This enables:

\begin{itemize}[leftmargin=*]
    \item Horizontal scaling without filesystem synchronization issues
    \item Media Library interface for drag-and-drop uploads
    \item Asset previews and organization by experiment
    \item One-click copy of asset references for YAML configuration
\end{itemize}

%========================================
\section{Data Export and Analysis}
%========================================

Research value lies in analysis, not data collection. B-IRES facilitates the transition from raw data to insights with comprehensive export capabilities.

\subsection{Export Formats}

\begin{itemize}[leftmargin=*]
    \item \textbf{Wide Format (CSV)}: One row per participant, one column per variable---ideal for SPSS, R, Stata, and Python pandas. Column names follow a hierarchical pattern (\texttt{stage\_question}) for easy identification.
    
    \item \textbf{Long Format (CSV/JSON)}: One row per event---for detailed timeline analysis, event sequence mining, and response latency studies.
    
    \item \textbf{JSON Lines}: Full hierarchical data with all metadata, suitable for programmatic processing and data pipelines.
    
    \item \textbf{SPSS-Compatible}: Wide format with variable labels and value labels embedded for direct import into SPSS.
\end{itemize}

\subsection{Wide Format Export Example}

A typical wide-format export might look like:

\begin{lstlisting}
participant_id,session_id,start_time,end_time,duration_sec,
  demographics_age,demographics_gender,demographics_education,
  task1_score,task1_duration_ms,task1_errors,
  likert_q1,likert_q2,likert_q3,
  feedback_comment,completed
  
p_abc123,s_xyz789,2026-01-04T10:00:00Z,2026-01-04T10:15:32Z,932,
  28,female,masters,
  85,45230,2,
  5,4,6,
  "Very engaging study",true
  
p_def456,s_uvw012,2026-01-04T10:02:15Z,2026-01-04T10:18:45Z,990,
  35,male,bachelors,
  72,52100,4,
  4,3,5,
  "Instructions were clear",true
\end{lstlisting}

\subsection{Long Format Export Example}

For event-level analysis:

\begin{lstlisting}
event_id,participant_id,session_id,timestamp,stage_id,
  event_type,element_id,value,duration_ms

evt_001,p_abc123,s_xyz789,2026-01-04T10:00:05Z,consent,
  view,consent_text,,
evt_002,p_abc123,s_xyz789,2026-01-04T10:00:35Z,consent,
  click,agree_checkbox,true,
evt_003,p_abc123,s_xyz789,2026-01-04T10:00:37Z,consent,
  submit,next_button,,30200
evt_004,p_abc123,s_xyz789,2026-01-04T10:00:38Z,demographics,
  view,demographics_form,,
evt_005,p_abc123,s_xyz789,2026-01-04T10:00:45Z,demographics,
  input,age_field,28,
evt_006,p_abc123,s_xyz789,2026-01-04T10:00:48Z,demographics,
  select,gender_field,female,
\end{lstlisting}

\subsection{Export API Examples}

Researchers can request exports programmatically or via the dashboard:

\begin{lstlisting}
# Wide format with all stages
GET /api/export/exp_abc123?format=wide

# Wide format with specific stages only
GET /api/export/exp_abc123?format=wide&stages=demographics,task1,feedback

# Long format for timeline analysis
GET /api/export/exp_abc123?format=long&event_types=click,submit

# JSON with full metadata
GET /api/export/exp_abc123?format=json&include_metadata=true

# Filter by date range
GET /api/export/exp_abc123?format=wide&from=2026-01-01&to=2026-01-31

# Filter by completion status
GET /api/export/exp_abc123?format=wide&completed=true
\end{lstlisting}

\subsection{Real-Time Monitoring Dashboard}

Researchers don't need to wait until data collection ends to assess progress:

\begin{itemize}[leftmargin=*]
    \item \textbf{Live Participant Counts}: Total started, in-progress, and completed by condition
    \item \textbf{Quota Status}: Visual indicators showing slots remaining for quota-limited conditions
    \item \textbf{Completion Funnel}: Drop-off rates at each stage with visual funnel chart
    \item \textbf{Response Time Distribution}: Histograms and percentiles for each stage
    \item \textbf{Active Sessions}: Currently active participants with stage and duration
    \item \textbf{Error Log}: Failed submissions, validation errors, and system issues
    \item \textbf{Anomaly Alerts}: Unusually fast completions, repeated IP addresses, failed attention checks
\end{itemize}

\subsection{Analytics Summary Example}

The dashboard provides at-a-glance metrics:

\begin{center}
\begin{tabular}{lrrr}
\toprule
\textbf{Metric} & \textbf{Condition A} & \textbf{Condition B} & \textbf{Total} \\
\midrule
Started & 156 & 148 & 304 \\
Completed & 142 & 135 & 277 \\
Completion Rate & 91.0\% & 91.2\% & 91.1\% \\
Median Duration & 12m 34s & 13m 02s & 12m 48s \\
Attention Check Pass & 138 & 131 & 269 \\
Data Quality Rate & 97.2\% & 97.0\% & 97.1\% \\
\bottomrule
\end{tabular}
\end{center}

%========================================
\section{Security and Ethics Considerations}
%========================================

Research involving human participants requires careful attention to data protection. B-IRES incorporates:

\begin{itemize}[leftmargin=*]
    \item \textbf{Local Data Storage}: All data remains on the researcher's infrastructure (no cloud dependencies unless explicitly chosen)
    \item \textbf{Pseudonymization}: Participant identifiers are randomly generated UUIDs with no personally identifiable information
    \item \textbf{Consent Documentation}: Built-in support for informed consent flows with configurable forced reading times
    \item \textbf{Data Minimization}: Researchers configure exactly what is collected; no unnecessary tracking
    \item \textbf{Role-Based Access Control}: Three roles (Admin, Researcher, Viewer) with appropriate permissions for multi-investigator projects
    \item \textbf{Variable Namespacing}: Server-side secrets (API keys, webhook URLs) are never exposed to the browser; only explicitly public variables are sent to clients
    \item \textbf{JWT Authentication}: Secure, stateless authentication with refresh tokens for admin access
\end{itemize}

%========================================
\section{User Management and Collaboration}
%========================================

B-IRES supports collaborative research teams with role-based access:

\begin{itemize}[leftmargin=*]
    \item \textbf{Admin}: Full platform access---manage users, all experiments, system configuration, and view all logs
    \item \textbf{Researcher}: Create and edit own experiments, view own experiment data, export own results
    \item \textbf{Viewer}: Read-only access to assigned experiments and their data (useful for research assistants or external collaborators)
\end{itemize}

Experiment ownership and permissions ensure that researchers can work independently while admins maintain oversight.

%========================================
\section{Extensibility}
%========================================

While B-IRES is designed to work out-of-the-box for common scenarios, the architecture supports extension:

\begin{itemize}[leftmargin=*]
    \item \textbf{Custom Block Types}: Developers can create new sub-stage types for specialized needs following the standard communication interface
    \item \textbf{Additional Storage Backends}: The pluggable adapter architecture supports adding PostgreSQL, MySQL, or other databases alongside MongoDB
    \item \textbf{Semantic Themes}: Institution-specific visual themes can be added to the theme library
    \item \textbf{Analysis Pipelines}: Webhook notifications and API access enable integration with automated analysis workflows
    \item \textbf{Internationalization}: Multi-language support for experiment content and UI labels
\end{itemize}

%========================================
\section{Conclusion}
%========================================

The \textbf{Bar-Ilan Research Evaluation System (B-IRES)} represents a significant step toward democratizing research technology for human-centric evaluations. By abstracting away technical complexity while maintaining research-grade rigor, B-IRES enables researchers across disciplines to design, deploy, and manage sophisticated multi-modal research trials.

Key advantages include:

\begin{itemize}[leftmargin=*]
    \item \textbf{Reduced Time-to-Launch}: From concept to live study in hours, not weeks
    \item \textbf{Lower Technical Barriers}: No programming required for standard use cases
    \item \textbf{Improved Data Quality}: Comprehensive logging captures what traditional surveys miss; attention checks ensure participant engagement
    \item \textbf{Scalable Collection}: Handle large samples with concurrent participant support and atomic quota management
    \item \textbf{Reproducible Research}: Configuration files serve as executable documentation; published experiments are immutable snapshots
    \item \textbf{Offline Resilience}: Participants can continue during network interruptions with automatic synchronization
    \item \textbf{Collaborative Teams}: Role-based access enables secure multi-investigator projects
\end{itemize}

As research involving human participants continues to evolve toward larger samples, more complex designs, and richer data types, platforms like B-IRES will become essential infrastructure for the academic and industry research community.

\vspace{2em}
\hrule
\vspace{1em}

\begin{center}
\textit{For technical specifications, deployment instructions, and configuration references, please consult the accompanying technical documentation.}
\end{center}

\end{document}
